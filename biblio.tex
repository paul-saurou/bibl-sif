\documentclass[11pt]{sdm}
\usepackage{graphicx}
\usepackage[style=authoryear, backend=biber]{biblatex}
\addbibresource{Biblio.bib}

%numeroter les pages
\pagestyle{plain}

\title{SCARE for Hardware SPN}
\author{First\_Name \textsc{Name}}
\supervisorOne{First\_Name \textsc{Name} of your first supervisor}
\supervisorTwo{First\_Name \textsc{Name} of your second supervisor}
\team{Name of the team in which you are doing your internship}
%One of:
% ens-Rennes  esir    insa-rennes   rennes1  
% enssat    logoUbs   tsupelec
%here rennes1 for example
\school{supelec}


% the domain should be one or two of:
% Technology for Human Learning 
% Artificial Intelligence 
% Computer Arithmetic
% Hardware Architecture
% Automatic Control Engineering
% Bioinformatics 
% Biotechnology
% Computational Complexity 
% Computational Engineering, Finance, and Science
% Computational Geometry 
% Computation and Language 
% Cryptography and Security 
% Computer Vision and Pattern Recognition
% Computers and Society 
% Databases 
% Distributed, Parallel, and Cluster Computing 
% Digital Libraries
% Discrete Mathematics 
% Data Structures and Algorithms 
% Embedded Systems 
% Emerging Technologies 
% Formal Languages and Automata Theory 
% General Literature 
% Graphics 
% Computer Science and Game Theory 
% Human-Computer Interaction 
% Computer Aided Engineering 
% Medical Imaging 
% Information Retrieval 
% Information Theory 
% Ubiquitous Computing 
% Machine Learning
% Logic in Computer Science 
% Multiagent Systems 
% Mobile Computing
% Multimedia
% Modeling and Simulation 
% Mathematical Software 
% Numerical Analysis 
% Neural and Evolutionary Computing 
% Networking and Internet Architecture 
% Operating Systems 
% Performance 
% Programming Languages 
% Robotics 
% Operations Research
% Symbolic Computation 
% Sound
% Software Engineering 
% Social and Information Networks 
% Systems and Control 
% Image Processing 
% Signal and Image Processing 
% Document and Text Processing
% Web
\domain{Domain: Cryptography and Security}

%write your abstract here
\abstract{write your abstract here}



\begin{document}
\maketitle

%*****************************************************************%

\section*{Introduction}

Here start your document - Should be 15 pages long \\

Following Kerchoffs' principle, the security of a cryptographic device shall not rely on the secrecy of its mechanisms \cite{Kerckhoffs_1883}.
However in some specific contexts, having a secret implementation can add a layer of security, by increasing the practical difficulty of the attack.

\section{State of the art of Side Channel Attacks on SPNs}

\subsection{SPNs, Feistel schemes, DES and AES}

Feistel scheme, DES

\cite{Standards2001}

\subsection{Side Channel Analysis classic attacks}

SPA, DPA, CPA explanation (ref papers ?)

\cite{Chari_Rao_Rohatgi_2003} presents template attacks, "the strongest form of side channel attack possible in an information theoretic sense".

\cite{Prouff_Rivain_1970} presents the theory of Mutual Information Attacks (MIA) in side channels.

Machine Learning is very popular right now.

\section{SCARE attacks}

\subsection{SCARE attacks on non AES ciphers}

\cite{Novak_2003} presents a side-channel attack on substitution blocks with a demonstration on a SIM card using COMP-128 cipher.

\cite{Daudigny_Ledig_Muller_Valette_2005} presents a SCARE attack on DES and propose new methods to exploit the power measurement information.

\cite{Guilley_Sauvage_Micolod_Réal_Valette_2010} presents two SCARE attacks on the parameters of a LFSR and DES.

\subsection{SCARE attacks on AES-like ciphers}

\cite{Tiessen_Knudsen_Kölbl_Lauridsen_2015} presents an integral cryptanalysis of an AES with a secret S-box and less rounds. It is not a SCA but is still closely related to our subject.

\cite{Rivain_Roche_2013} presents a generic SCARE attack against a wide class of SPN block ciphers.

FIRE (injection fault attempts) and SCARE attacks to recover the full set of secret parameters of an AES-like software implementation, even with masking and shuffling \cite{Clavier_Isorez_Marion_Wurcker_2015}.

\section{Application to hardware implementations}

\cite{Réal_Dubois_Guilloux_Valette_Drissi_2008} presents a SCARE attack on a general Feistel scheme with an hardware design.

SAKURA board reference

SCA attacks on FPGAs (\cite{Peeters_Standaert_Donckers_Quisquater_2005} or more relevantly \cite{Standaert_Ors_Preneel_2004} or something more recent)

\section{Conclusion}

\section*{References}

\printbibliography

\end{document}
%%% Local Variables:
%%% mode: latex
%%% TeX-master: t
%%% End: